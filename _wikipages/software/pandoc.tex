\documentclass[]{article}
\usepackage{lmodern}
\usepackage{amssymb,amsmath}
\usepackage{ifxetex,ifluatex}
\usepackage{fixltx2e} % provides \textsubscript
\ifnum 0\ifxetex 1\fi\ifluatex 1\fi=0 % if pdftex
  \usepackage[T1]{fontenc}
  \usepackage[utf8]{inputenc}
\else % if luatex or xelatex
  \ifxetex
    \usepackage{mathspec}
  \else
    \usepackage{fontspec}
  \fi
  \defaultfontfeatures{Ligatures=TeX,Scale=MatchLowercase}
\fi
% use upquote if available, for straight quotes in verbatim environments
\IfFileExists{upquote.sty}{\usepackage{upquote}}{}
% use microtype if available
\IfFileExists{microtype.sty}{%
\usepackage{microtype}
\UseMicrotypeSet[protrusion]{basicmath} % disable protrusion for tt fonts
}{}
\usepackage{hyperref}
\hypersetup{unicode=true,
            pdftitle={Pandoc Text Converter},
            pdfborder={0 0 0},
            breaklinks=true}
\urlstyle{same}  % don't use monospace font for urls
\IfFileExists{parskip.sty}{%
\usepackage{parskip}
}{% else
\setlength{\parindent}{0pt}
\setlength{\parskip}{6pt plus 2pt minus 1pt}
}
\setlength{\emergencystretch}{3em}  % prevent overfull lines
\providecommand{\tightlist}{%
  \setlength{\itemsep}{0pt}\setlength{\parskip}{0pt}}
\setcounter{secnumdepth}{0}
% Redefines (sub)paragraphs to behave more like sections
\ifx\paragraph\undefined\else
\let\oldparagraph\paragraph
\renewcommand{\paragraph}[1]{\oldparagraph{#1}\mbox{}}
\fi
\ifx\subparagraph\undefined\else
\let\oldsubparagraph\subparagraph
\renewcommand{\subparagraph}[1]{\oldsubparagraph{#1}\mbox{}}
\fi

\title{Pandoc Text Converter}
\date{}

\begin{document}
\maketitle

\subsubsection{Overview}\label{overview}

\href{http://pandoc.org/}{Pandoc} is a command line tool for
interconverting between markup languages. Pandoc supports common
languages like HTML, Markdown, and LaTeX, common document formats like
docx and epub, and lots of weird stuff I've never heard of before.

\[
R_{\mu \nu} - {1 \over 2}g_{\mu \nu}\,R + g_{\mu \nu} \Lambda
= {8 \pi G \over c^4} T_{\mu \nu}
\]

For example, Pandoc can convert Markdown with inline \(\LaTeX\) into a
PDF:

\begin{verbatim}
pandoc pandoc.md -f markdown -t latex -s -o pandoc.pdf
\end{verbatim}

You can generate an intermediate .tex file, customize it, then
\href{\%7B\%7B\%20site.baseurl\%20\%7D\%7D/files/pandoc.pdf}{convert it
to a PDF} in a separate step:

\begin{verbatim}
pandoc pandoc.md -f markdown -t latex -s -o pandoc.tex
pdflatex pandoc.tex
\end{verbatim}

To roll right through errors and clean up the \(\LaTeX\) build files
after generating the PDF, I created this script:

\begin{verbatim}
#!/bin/bash
mymd="$1"
mytex="$(echo "$mymd" | sed s/.md/.tex/)"
pandoc "$mymd" -s -f markdown -t latex -o "$mytex"
latexmk -pdf -f -interaction=nonstopmode "$mytex"
latexmk -pdf -f -c
\end{verbatim}

\end{document}
